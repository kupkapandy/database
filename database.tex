\documentclass[12pt, a4paper]{article}
\usepackage[utf8]{inputenc}
\usepackage[T1]{fontenc}
\usepackage{graphicx}

\setlength{\parindent}{0pt}

\begin{document}
\tableofcontents
\newpage
\section{Normalizacja - Teoria}
\subsection*{Co to jest normalizacja?}
\textbf{Normalizacja} to proces sprowadzania schematu do odpowiedniej postaci, w gównej mierze polega
na podzielenie tabeli i połączenie ich kluczem głównym. Wady (anomalie) które mogą wystąpić podczas normalizacji:

\subsection*{Atrybuty kluczowe i niekluczowe}
W procesie normalizacji musimy zaznaczyć atrybuty kluczowe (czyli atrybuty posiadające klucz),
atrybuty niekluczowe (nie należą do żadnego klucza), Atrybuty kluczowe trzeba podkreślić.

\subsection*{Anomalie występujące w normalizacji}
\begin{itemize}
  \item \textbf{Dołączania} - Nie można dołączyć rekordu, ponieważ nie znamy jego innych danych.
  \item \textbf{Aktualizacji} - W jednym miejscu dane są aktualizowane, a w drugim nie.
  \item \textbf{Usuwania} - Usunięcie rekordu spowoduje usunięcie danych innego rodzaju.
\end{itemize}

\subsection*{Trzy kluczowe zasady normalizacji}
W procesie normalizacji są 3 własności które trzeba zachować:
\begin{itemize}
  \item Żaden atrybut nie zostanie utracony podczas procesu normalizacji
  \item Zmiana tabeli (czyli je podzielenie etc.) nie prowadzi do utraty informacji
  \item Każda zależność funkcyjna posiada własną tabele
\end{itemize}
\subsection*{Pierwsza postać normalna 1NF}
Schemat relacyjny jest w 1NF, gdy wszystkie kolumny posiadają wartości atomowe.
\subsection*{Druga postać normalna 2NF}
Schemat relacyjny jest w 2NF, gdy jest w 1NF i kiedy atrybuty niekluczowe są
w pełni zależne od całego klucza głównego.
\subsection*{Trzecia postać normalna 3NF}
Schemat relacyjny jest w 3NF, gdy jest w 2NF i atrybuty niekluczowe są niezależne od innych
atrybutów niekluczowych.
\end{document}
