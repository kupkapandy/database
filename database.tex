\documentclass[12pt, a4paper]{article}
\usepackage[utf8]{inputenc}
\usepackage[T1]{fontenc}
\usepackage{graphicx}
\usepackage{amsmath}
\usepackage{textcomp}

\setlength{\parindent}{0pt}

\begin{document}
\tableofcontents
\newpage
\section{Normalizacja - Teoria}
\subsection*{Co to jest normalizacja?}
\textbf{Normalizacja} to proces sprowadzania schematu do odpowiedniej postaci, w gównej mierze polega
na podzielenie tabeli i połączenie ich kluczem głównym.

\subsection*{Atrybuty kluczowe i niekluczowe}
W procesie normalizacji musimy zaznaczyć atrybuty kluczowe (czyli atrybuty posiadające klucz),
atrybuty niekluczowe (nie należą do żadnego klucza), Atrybuty kluczowe trzeba podkreślić.

\subsection*{Anomalie występujące w normalizacji}
\begin{itemize}
  \item \textbf{Dołączania} - Nie można dołączyć rekordu, ponieważ nie znamy jego innych danych.
  \item \textbf{Aktualizacji} - W jednym miejscu dane są aktualizowane, a w drugim nie.
  \item \textbf{Usuwania} - Usunięcie rekordu spowoduje usunięcie danych innego rodzaju.
\end{itemize}

\subsection*{Trzy kluczowe zasady normalizacji}
W procesie normalizacji są 3 własności które trzeba zachować:
\begin{itemize}
  \item Żaden atrybut nie zostanie utracony podczas procesu normalizacji
  \item Zmiana tabeli (czyli je podzielenie etc.) nie prowadzi do utraty informacji
  \item Każda zależność funkcyjna posiada własną tabele
\end{itemize}
\subsection*{Pierwsza postać normalna 1NF}
Schemat relacyjny jest w 1NF, gdy wszystkie kolumny posiadają wartości atomowe.
\subsection*{Druga postać normalna 2NF}
Schemat relacyjny jest w 2NF, gdy jest w 1NF i kiedy atrybuty niekluczowe są
w pełni zależne od całego klucza głównego.
\subsection*{Trzecia postać normalna 3NF}
Schemat relacyjny jest w 3NF, gdy jest w 2NF i atrybuty niekluczowe są niezależne od innych
atrybutów niekluczowych.
\section{Wyrażenia, maski i reguły poprawności}
\subsection*{Podstawowe wyrażenia}
\textbf{Identyfikatory} - To nazwy używane do odwoływania się do elementów w bazie danych takie jak np.
\begin{itemize}
  \item \textbf{Pola tabeli}
  \item \textbf{Formanty}
  \item \textbf{Właściowści}
\end{itemize}
Dzęki identyfikatorą można tworzyć wyrażenia które odwołują się do wartości elementów np.
$= [\text{Tabela}].[\text{Pole z tabeli}]$
\vspace{1em}

\textbf{Stałe} - Czyli wartości podobne do \textbf{const}, najczęściej to true, false, NULL.
\subsection*{Podstawowe wartości, funkcje, operatory}
\textbf{Wartości}:
\begin{itemize}
  \item \textbf{Liczby} np. $1, -234$
  \item \textbf{String} (Ciąg tekstowy) ujęty w $""$ lub $''$ na przykład "Mężczyzna".
  \item \textbf{Data/Godzina} (Ujęte w znaki \#), np. \#03-07-2007\#, \#07-Mar-07\#
\end{itemize}
\vspace{1em}

\textbf{Funkcje}
\begin{itemize}
  \item \textbf{YEAR(data)} - Zwraca rok z daty
  \item \textbf{LEN("string")} - Zwraca długość stringa
  \item \textbf{LCase(string)} - Zwraca string przekonwertowany na małe litery
\end{itemize}
\vspace{1em}

\textbf{Operatory}
\begin{itemize}
  \item \textbf{$+, -, *, /$} – podstawowe operatory
  \item \textbf{$\backslash$} – dzielenie i zaokrąglenie do liczby całkowitej
  \item \textbf{MOD}, \^{} – modulo i podniesienie do potęgi
  \item \textbf{$<, <=, >, >=, =, <>$} - oznacza różne (Reszta jest wiadoma)
  \item \textbf{$+$} - Łączy stringi, jeśli jeden string to NULL, wynik to NULL
  \item \textbf{$\&$} - Łączy stringi, jeśli jeden string to NULL, wynik to drugi string
  \item \textbf{NOT, AND, OR, XOR} - Operatory logiczne, zwracają wynik w postaci logicznej
  \item \textbf{IS NULL} albo \textbf{IS NOT NULL} - Porównanie wartości z NULL
  \item \textbf{LIKE "przykład"} - Porównuje string z podanym przykładem, w wzorcu mogą zostać użyte symbole wieloznaczne np. $*, ?$.
    LIKE "a*" - Pokaże wartości pól które zaczynają się od litery "a", lub "*a" pokaż pola na końcu z literą "a".
  \item \textbf{BETWEEN var1 AND var2} - Sprawdza czy wartośc jest pomiędzy var1 i var2 (działa też na daty)
  \item \textbf{IN ("var1", "var2", \dots)} - Sprawdza czy wartość jest w zbiorze wartości
\end{itemize}
\subsection*{Maski}
\begin{itemize}
  \item \textbf{0} – Cyfra, musi być wpisana (brak znaków + i -).
  \item \textbf{9} – Cyfra lub spacja, nie musi być wpisana (brak znaków + i -).
  \item \textbf{\#} – Cyfra lub spacja, nie musi być wpisana (może być znak + lub -).
  \item \textbf{L} – Litera, musi być wpisana.
  \item \textbf{?} – Litera, nie musi być wpisana.
  \item \textbf{A} – Litera lub cyfra, musi być wpisana.
  \item \textbf{a} – Litera lub cyfra, nie musi być wpisana.
  \item \textbf{\&} – Dowolny znak, musi być wpisany.
  \item \textbf{<} – Zamienia litery na małe.
  \item \textbf{>} – Zamienia litery na duże.
  \item \textbf{!} – Wymusza wypełnienie maski od lewej do prawej.
  \item \textbf{\textbackslash} – Następny znak nie jest kodem maski, ale zwykłym znakiem.
\end{itemize}
Przykłady masek:
\begin{itemize}
  \item Data - "00-00-00" - Musi być wpisana liczba
  \item Data - "90-90-00" - Liczba 9 może być pusta
  \item Nazwisko - ">L<???????????????" - Nazwisko zaczyna się z dużej litery, reszta to małe litery, max. 16 liter.
\end{itemize}
\subsection*{Reguła poprawności}
\textbf{Reguła poprawności} - Określamy zakres danych jakie można wprowadzić do danego pola, np.
\begin{itemize}
  \item >=50 AND <=1500 (Wartości tylko z zakresu od 50 do 1500)
  \item >=\#1997-01-01\# and <=\#1998-1-1\# Dla dat od 1997-01-01 do 1998-01-01
  \item
\end{itemize}
\end{document}
